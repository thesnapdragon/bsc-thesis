%----------------------------------------------------------------------------
% Abstract in hungarian
%----------------------------------------------------------------------------
\chapter*{Kivonat}\addcontentsline{toc}{chapter}{Kivonat}

A szemi-struktúrált adatok felhasználása rendkívül népszerű a mesterséges intelligencia, természetes nyelvi feldolgozás, de az informatika más területein is. Az ilyen szemi-struktúrált adatforrásokat, mint például a legnagyobb kollaboratívan szerkeszthető tudásbázist, a Wikipédiát felhasználó megoldások száma mégis igen kevés. Szakdolgozatom célja egy kutatók számára is használható rendszer tervezésének és fejlesztésének bemutatása, mely a Wikipédiát dolgozza fel, és teszi elérhetővé kutatások céljából. A rendszer egy feldolgozólánc formájában fog megvalósulni, mely az OSGi keretrendszert használva, annak flexibilitását kihasználva egyszerűen kiegészíthető, illetve továbbfejleszthető lesz, így alkalmassá válik további kutatások kiindulópontjaként szolgálni.

A tervezés előtt mindenképp meg kell vizsgálni a hasonló megoldásokat is, hiszen azok tapasztalatait érdemes felhasználni: hibáikból tanulni lehet, a jól bevált megoldásokat pedig meg kell fontolni, hogy beépíthetőek-e az alkalmazás architektúrájába. Az elérhető szemi-struktúrált adatforrásokat feldolgozó alkalmazásokat végigvizsgálva mindegyikben egy alapvető hiányosság tűnt fel: egyik sem elég flexibilis, és a legtöbb program csak egy nagyon speciális célra alkalmas, így általános kutatások alapjául nem használhatóak. A megoldások közös jellemzője, hogy alapvetően három jól meghatározott, nagyobb előfeldolgozó komponenst tartalmaznak, így ezeket a komponenseket (információ gyűjtése, feldolgozása, eltárolása) terveztem meg és implementáltam én is a követelményeknek megfelelő módon.

A rendszer architektúráját nagyban meghatározza a flexibilitást és karbantarthatóságot eredményező OSGi technológia, melynek komponens modellje egy megszokottól eltérő Java nyelvű fejlesztést.

A tényleges tervezésnél a különálló OSGi komponenseket külön-külön készítettem el, figyelve arra, hogy az alkalmazás teljesítménye minél nagyobb legyen. A gyorsaságot legnagyobb részt a többszálú futásnak köszönheti az alkalmazás, de a feldolgozás több részének aszinkron kivitelezése és nagy teljesítményű technológiák alkalmazása is sokat segített.

Az implementációs fázisban ismertetem a kipróbált technológiákat, melyiknek milyen előnyei, hátrányai vannak, és miért esett rájuk a választás. A technológiai elemeken kívül a tervezési minták használata is hangsúlyos az elkészített munkámban.

A kivitelezés eredményeit végül egy mérésen keresztül bemutatom, mely fázis átvezet a továbbfejlesztési lehetőségekhez. Itt mérésekkel alátámasztva mutatom be, melyik részt, hogyan lehetne továbbfejleszteni a specifikációtól eltérve, hogy a követelményeknek megfelelően, de mégis egy hiánypótló alkalmazás szülessen a szemi-struktúrált feldolgozórendszerek körében.

\vfill

%----------------------------------------------------------------------------
% Abstract in english
%----------------------------------------------------------------------------
\chapter*{Abstract}\addcontentsline{toc}{chapter}{Abstract}

Using of semi-structured datas is very popular in the field of artificial intelligence (AI), natural language processing (NLP) but other branch of computer sciences too. Nevertheless, there are only a few solution that are using semi-structured datasources, for example the greatest collaboratively editable knowledge base, the Wikipedia. My thesis aims to present the full lifecycle of a new research tool, that can help in computer science researches, and it is able to process the articles of Wikipedia and publishes the result for the researchers.

Before the design similar solutions have to investigated, because from their experience we can learn a lot. We can avoid common mistakes, but use for example a good architecture. Examining the available solutions it can be see, that none of them is flexible and they can be used only at specific tasks. Common property of the applications that they all have three major, well defined components (gathering, processing and storing the datas). These three components were designed and implemented in my thesis regarding the specified requirements.

A system's architecture based on OSGi module system and service platform that comes with flexibility and maintainability. The OSGi framework extends Java language's ability with a component driven development method.

At the design phase the separate OSGi components were created one-by-one taking care of high performance. The speed of the application comes from the multi threaded approach, but the async implementation and the usage of high performance technologies bring a lot of speed too.

At the implementation chapter the tested technologies are demonstrated, covering the advantages and disadvantages of each technology. Besides the technical elements, the design patterns are very important in my thesis too.

The results of the implementation will be demonstrated through a measurement, which chapter leads to opportunities of further development. The further development ideas are proved with measurements and it is explained how is it possible to modify the application to create a useful application in the field of semi-structured data processing systems.

\vfill