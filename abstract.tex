%----------------------------------------------------------------------------
% Abstract in hungarian
%----------------------------------------------------------------------------
\chapter*{Kivonat}\addcontentsline{toc}{chapter}{Kivonat}

A szemi-struktúrált adatok felhasználása rendkívül népszerű a mesterséges intelligencia, természetes nyelvi feldolgozás, de az informatika más területein is. Az ilyen szemi-struktúrált adatforrásokat, mint például a legnagyobb kollaboratívan szerkeszthető tudásbázist, a Wikipédiát felhasználó megoldások száma mégis igen kevés. Szakdolgozatom célja egy kutatók számára is használható rendszer tervezésének és fejlesztésének bemutatása, mely a Wikipédiát dolgozza fel, és teszi elérhetővé kutatások céljából. A rendszer egy feldolgozólánc formájában fog megvalósulni, mely az OSGi keretrendszert használva, flexibilitásának köszönhetően egyszerűen kiegészíthető, illetve továbbfejleszthető lesz, így alkalmassá válik további kutatások kiindulópontjaként szolgálni.

A tervezés előtt mindenképp meg kell vizsgálni a hasonló megoldásokat is, hiszen azok tapasztalatait érdemes felhasználni: hibáikból tanulni lehet, a jól bevált megoldásokat pedig meg kell fontolni, hogy beépíthetőek az alkalmazás architektúrájába. Az elérhető szemi-struktúrált adatforrásokat feldolgozó alkalmazásokat végigvizsgálva mindegyikben egy alapvető hiányosság tűnt fel, egyik sem elég flexibilis, és a legtöbb program csak egy nagyon speciális célra alkalmas, így általános kutatások alapjául nem használhatóak. Ezt a tulajdonságot leszámítva azonban alapvetően három nagyobb előfeldolgozó komponenst tartalmaznak ezek a rendszerek, így ezeket a komponenseket (információ gyűjtése, feldolgozása, eltárolása) terveztem meg és implementáltam én is a követelményeknek megfelelő módon.

A rendszer architektúráját nagyban meghatározza a flexibilitás és karbantarthatóságot eredményező OSGi technológia, melynek komponens modelje egy megszokottól eltérő Java nyelvű fejlesztést tesz lehetővé, így a tervezés előtt megismertem az OSGi platform lehetőségeit.

A tényleges tervezésnél részletesen megterveztem a különálló OSGi komponenseket, figyelve arra, hogy az alkalmazás teljesítménye minél nagyobb legyen. A gyorsaságot legnagyobb részt a többszálú futásnak köszönheti az alkalmazás, de a feldolgozás több részének aszinkron kivitelezése is sokat segített.

Az implementációs fázisban ismertetem a kipróbált technológiákat, melyiknek milyen előnyei, hátrányai vannak, és miért esett rájuk a választás. A technológiai elemeken kívül a tervezési minták is nagy hangsúlyt kapnak a dolgozatomban.

A kivitelezés eredményeit végül egy mérésen keresztül bemutatom, mely rész átvezet a továbbfejlesztési lehetőségekhez. Itt mérésekkel alátámasztva mutatom be, melyik részt, hogyan lehetne továbbfejleszteni a specifikációtól eltérve, hogy a követelményeknek megfelelően, de egy hiánypótló alkalmazás szülessen a szemi-struktúrált feldolgozórendszerek körében.

\vfill

%----------------------------------------------------------------------------
% Abstract in english
%----------------------------------------------------------------------------
\chapter*{Abstract}\addcontentsline{toc}{chapter}{Abstract}

%% TODO

\vfill