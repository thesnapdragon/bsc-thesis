%----------------------------------------------------------------------------
% Bevezető
%----------------------------------------------------------------------------

\chapter*{Bevezető}\addcontentsline{toc}{chapter}{Bevezető}
\label{cha:intro}

Mai világunkban a tudás a legnagyobb érték. A tudás valójában kontextusba ágyazott információ, mely elemi adatokból épül fel. Hosszú ideje irányulnak kutatások és fejlesztések az informatikában a tudás minél hatékonyabb megszerzésére, azonban, ha egy nehéz és összetett problémát akarunk megoldani valamilyen módszerrel, azt tapasztaljuk, hogy a hozzá szükséges tudás megszerzése lesz mindig a szűk keresztmetszet (ez az ún. \textit{knowledge acquisition bottleneck}).

Ez a helyzet egy ideje a különféle adatábrázolási módszereknek köszönhetően kezdett megváltozni. A struktúrálatlan nehezen feldolgozható adatok mellett megjelentek a struktúrált és szemi-struktúrált adatok, melyekből az információt sokkal gyorsabban és könnyebben lehet kinyerni, és olyan automatizált folyamatokban is felhasználhatóak már, ahol korábban emberi közreműködés lett volna szükséges. Megjelentek a kollaboratívan szerkeszthető tudásbázisok, ezek közül is a legnépszerűbb és legnagyobb a Wikipedia, mely egy szemi-struktúrált információforrás. Ezen tudásbázisokat főleg a mesterséges intelligencia \cite{aijournal}, természetes nyelvi feldolgozás, valamint a számítógépes nyelvészet területén, de az informatika szinte minden ágában ugyanúgy felhasználják.

A szemi-struktúrált információforrások, mint a Wikipedia (továbbiak például a Flickr, Twitter vagy a Yahoo! Answers) egyesítik a struktúrált és struktúrálatlan források előnyeit, így rendkívül jó kiindulópontjai különféle kutatásoknak. A struktúrálatlan adatokkal szembeni előnye, hogy gépek számára is értelmezhető információt tárolnak; a struktúrált adatok előállításánál és kezelésénél pedig kevésbé erőforrás-igényes a szemi-struktúrált adatok létrehozása és karbantartása.

Az informatika ezen területén történő kutatások már 2005-2006 körül megkezdődtek, azonban az ezek eredményeit felhasználó fejlesztések száma nem túl sok. Célom tehát az eddig elkészült és elérhető fejlesztések vizsgálata, információk és tapasztalatok összesítése, valamint ezeket továbbfejlesztve egy korszerűbb rendszer összeállítása, melyre alapozva eredményes kutatásokat lehet kezdeményezni, a legnagyobb elérhető szemi-struktúrált erőforrás a Wikipedia segítségével, a lehető legflexibilisebb módon.

% chapter intro (end)